% LaTeX Curriculum Vitae Template
% Copyright (C) 2004-2009 Jason Blevins <jrblevin@sdf.lonestar.org>
% http://jblevins.org/projects/cv-template/

\PassOptionsToPackage{quiet}{fontspec}
\documentclass[letterpaper]{article}

\usepackage{hyperref,geometry,bookmark}
\usepackage[UTF8]{ctex}
\usepackage{fontenc} % Requires XeTeX
\usepackage{enumitem} % Enumerate with [resume] option

% 在加载sectsty包之前添加这行代码
\let\underbar\relax

\usepackage{sectsty} % Custom section fonts
\usepackage{graphicx}
\usepackage{wrapfig}

% Set your name here
\def\name{吉庆}

% Replace this with a link to your CV if you like, or set it empty
% (as in \def\footerlink{}) to remove the link in the footer:
\def\ftlink{http://llig.buaa.edu.cn/info/1017/1280.htm}

% Print the month and year
% \renewcommand{\today}{\ifcase \month \or January \or February \or March \or April \or May %
% \or June\or July\or August\or September\or October \or November\or December\fi%
% \space \number \year}
\renewcommand{\today}{\number\year 年 \number\month 月}

% The following metadata will show up in the PDF properties
\hypersetup{
  % colorlinks = true,
  % urlcolor = black,
  pdfauthor = {Qing Ji},
  pdfkeywords = {Economics; Econometrics; Empirical Industrial Organization; Applied Microeconomics},
  pdftitle = {Qing Ji, Curriculum Vitae},
  pdfsubject = {Curriculum Vitae},
  pdfpagemode = UseNone
}

\geometry{
  body={6.5in, 9.0in},
  left=1.0in,
  top=1.0in
}

% Customize page headers
\usepackage{fancyhdr}
\pagestyle{fancy}
\fancyhf{} % clear all header fields and footer fields
\newcommand{\helv}{\fontsize{9.5pt}{9.5pt}\selectfont} % 9.5号字体,单倍行间距
% \fancyhead[L]{\helv \name}
% \fancyhead[R]{\helv 更新日期:\today}
\fancyfoot[C]{\helv \thepage}
\renewcommand\headrulewidth{0pt}
\renewcommand\footrulewidth{0pt}
\thispagestyle{empty} % empty for first page

% Custom section fonts
\usepackage{sectsty}
\sectionfont{\rmfamily \mdseries \Large}
\subsectionfont{\rmfamily \mdseries \large}

% Other possible font commands include:
% \ttfamily for teletype,
% \sffamily for sans serif,
% \bfseries for bold,
% \scshape for small caps,
% \normalsize, \large, \Large, \LARGE sizes.

% Don't indent paragraphs.
\setlength\parindent{0in}

% Make lists without bullets
\renewenvironment{itemize}{
  \begin{list}{}{
    \setlength{\leftmargin}{0.2in}
    \setlength{\itemsep}{0.15in}
    \setlength{\parskip}{0.05in}
    \setlength{\parsep}{0.05in}
  }
}{
  \end{list}
}
\setlist[enumerate]{itemsep=0.15in}

\begin{document}

\begin{wrapfigure}{r}{0.45\textwidth}
  \vspace*{-0.1cm}
    \centering
      \includegraphics[width=0.17\textwidth]{jq.jpg}
	\vspace{-6cm}
\end{wrapfigure}

% Place name at left
{\huge \bf \name}
% \centerline{\huge \bf \name} % Alternatively, print name centered:
\bigskip

% \section*{联系信息}
\begin{minipage}[t]{0.4\linewidth}
  \centering
    \begin{tabular}{rl}
      面貌:& 中共党员 \\
      籍贯:& 陕西渭南 \\
      学校:& 北京航空航天大学 \\
      邮箱:& bhjiqing@buaa.edu.cn \\
      电话:& 156-0060-0929 \\
    \end{tabular}
\end{minipage}
\begin{minipage}[t]{0.45\linewidth}
  \centering
    \begin{tabular}{ll}
      % 性别:& 男 \\
      % 面貌:& 中共党员 \\
      % 电话:& (+86) 156-0060-0929 \\
      % 邮箱:& bhjiqing@buaa.edu.cn
    \end{tabular}
\end{minipage}

\section*{研究领域}
\hrule
\vspace{0.25cm}
  \begin{itemize}
    \item 低碳能源经济与政策;环境经济学;应用计量经济学;实证产业组织;大数据处理
  \end{itemize}

\section*{教育背景}
\hrule
\vspace{0.25cm}
  \begin{itemize}
    \item 博士(2018.09 - 2023.01):北京航空航天大学,经济管理学院,管理科学与工程
      \begin{itemize}
        \item 指导老师:\href{http://sem.buaa.edu.cn/szdw/yyjjx/fy/jbxx.htm}{范英教授}(北航经管学院院长、二级教授、教育部长江学者特聘教授等)\\
        指导老师:\href{https://shi.buaa.edu.cn/wangbuaa08/zh_CN/index.htm}{王楚男副教授}(北航青年拔尖人才)\\
        学位论文题目:《交通部门中政策工具的环境影响研究》
      \end{itemize}
    \item 博士(2018.09 - 2023.01,线上联培):北卡罗来纳州立大学,农业资源与经济系,实证产业组织
      \begin{itemize}
        \item 指导老师:\href{https://cals.ncsu.edu/agricultural-and-resource-economics/people/xiaoyong-zheng/}{郑霄勇教授}(美国农业经济学期刊副主编、农业与食品工业组织副主编等)\\
        研究内容:建立政策影响的综合评估结构模型,包含消费者、生产者、政府和社会外部性四部分
      \end{itemize}
    \item 硕士(2015.09 - 2017.06):中国石油大学(北京),经济管理学院,产业经济学
      \begin{itemize}
        \item 指导老师:\href{https://www.cup.edu.cn/sba/szdwgb/yyjjxgb/l2/177732.htm}{刘毅军教授}(中国石油学会天然气专业委员会经济学组委员、《天然气工业》编委等)\\
        研究内容:能源经济与政策(天然气)
      \end{itemize}
    \item 学士(2010.09 - 2014.06):西安石油大学,经济管理学院,工程管理
  \end{itemize}

\section*{学术成果}
\hrule
\vspace{0.25cm}
  \begin{enumerate}
    \item \textbf{Qing Ji}, Chunan Wang, Ying Fan*. Environmental and Welfare Effects of Vehicle Purchase Tax: Evidence from China[J]. {\it Energy Economics}, 2022, 115: 106377. (SSCI;2021;JCR Q1;ABS 3;Impact Factor*:9.252;CiteScore*:11.3) \\
    \textbf{研究内容}:本文通过构造结构型需求和供给模型并进行反事实分析,不仅评估了燃油汽车购置税税率改革的环境和福利效应,而且对比了油车购置税税率提高和电车购车补贴的福利效益
    \item \textbf{Qing Ji}, Chunan Wang*, Xiaoyong Zheng, Ying Fan. Quantifying the Welfare Effects of Battery Electric Vehicle Subsidies: Evidence from China[J]. {\it Journal of the Association of Environmental and Resource Economists}. \textbf{Revise and Resubmit}, 2022.(环境经济领域顶刊;SSCI;2021;JCR Q1;ABS 3;Impact Factor*:3.923;CiteScore*:7.8) \\
    \textbf{研究内容}:本文估计了一个结构型需求和供给模型以量化了电动汽车补贴对消费者福利、制造商利润、政府收入、温室气体 排放和空气污染物排放的环境外部因素以及交通事故和拥堵的非环境外部因素的影响。最重要的是,本文还将福利效应分解为两部分,一部分归因于基于驾驶里程的基准补贴,另一部分归因于针对技术创新的调整系数
    \item \textbf{Qing Ji}, Ying Fan, Chunan Wang*. Environmental and Social Welfare Impacts of Airline Liberalization: Evidence from the Central Asian - China Market. Working Paper, 2022.\\
    \textbf{研究内容}:本文采用了结构模型和反事实分析结合的方法,从局部均衡的角度调查了在2012年中亚-北京的航线的连通性分别类似于中亚-俄罗斯、中亚-前苏联国家和中亚-其他亚洲国家的航线连通性的三种情景中,环境外部成本和社会福利的变化情况
    \item \textbf{Qing Ji}, Ying Fan, Chunan Wang. Is the Opening of High-Speed Rail Beneficial to the Construction of Green Civil Aviation? A Case Study Based on the Beijing-Shanghai Market in China. Working Paper, 2022.\\
    \textbf{研究内容}:基于使用了MIDT数据库提供的京沪航空月度客运数据,本文估计了消费者需求偏好和航空公司的边际成本参数,并在局部均衡的视角下通过反事实实验量化了京沪高铁开通对其民航及整个交通市场的环境影响。这不仅为今后我国空铁产业布局的提供了重要参考,而且为我国交通部门实现“双碳目标”提供了政策支持
  \end{enumerate}

\section*{科研项目}
\hrule
\vspace{0.25cm}
  \begin{enumerate}
    \item “双碳目标下我国航空高铁竞争合作关系研究”,教育部“春晖计划”合作科研项目,2023 - 2024,项目核心成员 \\
    \textbf{项目职责}:主要承担了结构估计模型的构建,并分析高铁取代短途航线的环境影响研究
    \item “直飞航线开通与航空网络竞争研究:基于博弈模型和结构估计方法”,国家自然科学基金青年基金项目,2021 - 2023,项目核心成员 \\
    \textbf{项目职责}:主要承担了结构估计模型的构建,并分析了直飞航线开通的环境和社会福利影响
    \item “一带一路倡议下中国民航直飞航线开通潜力研究:基于结构估计方法和民航大数据”,北航经管学院航空航天专项启动经费项目,2021 - 2022,项目核心成员 \\
    \textbf{项目职责}:主要评估了推动中亚五国和中国双边航权开放的环境和社会福利影响
    \item “绿色导向的社会经济复杂系统管理决策”,国家自然科学基金创新研究群体项目,2020 - 2025,项目核心成员 \\
    \textbf{项目职责}:在补贴退坡的情况下,主要研究了提高油车辆购置税对促进电车销量的间接激励作用
    \item “能源体系变革的规律与驱动机制研究”,国家自然科学基金重大项目,2017 - 2021,项目核心成员 \\
    \textbf{项目职责}:主要评估了电车补贴对消费者福利、制造商利润、政府收入和社会外部性的影响
    \item “市场化定价对天然气市场的影响及中国石化应对策略研究”,中石化股份有限公司委托项目,2016 - 2017,项目核心成员 \\
    \textbf{项目职责}:结合我国的实际国情,主要调研了欧美天然气产业变革经验,并相应地提出我国天然气行业的应对策略
    \item “天然气产业链及其价格研究”,国家发展和改革委员会委托项目,2015 - 2016,项目核心成员 \\
    \textbf{项目职责}:主要进行了对美国天然气价格改革历程的分析,并在此基础上研究了北美天然气定价中心演变规律
  \end{enumerate}

\section*{学术交流与报告}
\hrule
\vspace{0.25cm}
  \begin{enumerate}
    \item 第一届全国碳中和博士生论坛 - 低零碳交通分会场,2022.12 \\
    \textbf{报告题目}:《中国电动汽车补贴政策的环境和社会福利影响》
    \item 第八届能源经济学术创意大赛北航赛区决赛,2022.03 \\
    \textbf{报告题目}:《航空自由化环境和社会福利影响:来自中亚 - 中国市场证据》
    \item The 24th Air Transport Research Society World Conference,2021.08 \\
    \textbf{报告题目}:《Environmental and Social Welfare Impacts of Airline Liberalization: Evidence from the Central Asian - China Market》
  \end{enumerate}

\section*{专业技能}
\hrule
\vspace{0.25cm}
\begin{enumerate}
  \item 擅长领域:
    \begin{itemize}
      \item 数据分析: Matlab、Stata和Python \\
            数据库管理: MySQL \\
            开发环境和版本控制: VScode和GitHub
    \end{itemize}
  \item 主要经历:
    \begin{itemize}
      \item  2.1. 气候综合评估模型的构建(Matlab)
      \begin{itemize}
        \item 使用Matlab构建能源经济综合评估模型 \\
              负责模型的设计和实现,并进行经济变量的分析、模拟和预测
      \end{itemize}
      \item  2.2. 国际民航组织 MIDT 数据处理(Stata)
      \begin{itemize}
        \item 使用Stata处理国际民航数据(MIDT)\\
              数据清洗、变量筛选、统计计算、可视化展示等操作
      \end{itemize}
      \item  2.3. 网络爬虫和数据库搭建(Python \& MySQL)
      \begin{itemize}
        \item 使用Python进行网络爬虫,从汽车之家网页获取动态加载数据 \\
              网页解析、数据抓取、数据清洗、MySQL数据库设计、表结构定义等操作
      \end{itemize}
    \end{itemize}
\end{enumerate}

% \section*{社会实践}
% \hrule
% \vspace{0.25cm}
%   \begin{enumerate}
%     \item 中国石油规划总院,管道部实习生,2016.09 - 2016.11 \\
%     \textbf{工作内容}:参与西气东输相关长输管道情况梳理,结合昆仑银行燃气贷业务的开展情况,对搭建天然气销售平台(B2B2C)的可行性进行了分析
%     \item 百度,在线管理部实习生,2016.05 - 2016.08 \\
%     \textbf{工作内容}:参与店铺人群特征统计分析项目,以百度舆情、百度司南、百度大数据+等大数据生态相关的产品为依托,根据用户画像服务对产品前端修改需求与相关功能迭代
%     \item 申万宏源证券总部,机构事业部实习生,2016.02 - 2016.04 \\
%     \textbf{工作内容}:参与对已有金融投资产品和策略进行量化数理分析,通过对营业销售业务数据进行监测归纳和整理,验证各种交易策略并进行可操作性研判
%     \item 北京西店志愿者服务中心,教学岗位志愿者,2015.09 - 2015.12 \\
%     \textbf{工作内容}:任职1 - 5年级班主任与数学老师
% \end{enumerate}

\section*{奖项与荣誉}
\hrule
\vspace{0.25cm}
  \begin{itemize}
    \item 2022年,第八届全国大学生能源经济学术创意大赛“斯伦贝谢杯”北航赛区二等奖
    \item 2018、2019、2020、2021年,北京航空航天大学学业奖学金
    \item 2016年,国家能源局2015年度能源软科学研究优秀成果奖
    \item 2015、2016年,中国石油大学(北京)学业奖学金
  \end{itemize}

\section*{学术兼职}
\hrule
\vspace{0.25cm}
  \begin{itemize}
    \item 中国系统工程学会能源资源系统工程分会会员
    \item 担任国际期刊《Frontiers in Energy Research》匿名审稿人
  \end{itemize}
\bigskip

% Footer
\begin{flushright}
  \begin{footnotesize}
    % Last updated: \today \\
    更新日期:\today \\
    % \href{\ftlink}{\ftlink}
  \end{footnotesize}
\end{flushright}

\end{document}